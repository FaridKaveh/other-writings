\documentclass[a4paper]{article}

\usepackage[T1]{fontenc}
\usepackage[utf8]{inputenc}
\usepackage{lmodern}
\usepackage{graphicx}
\usepackage[english]{babel}
\usepackage{csquotes}
\setlength\parindent{0pt}
\setlength\parskip{5pt}
\usepackage[notes,backend=biber]{biblatex-chicago}
\usepackage{ccpaper}
\bibliography{sample}


\begin{document}
\title{Why is the Unites States a Nation?}
\subtitle{}
\author{Farid Kaveh}

\maketitle

Why did the thirteen colonies come together to declare their independence from the crown? And why was this original union maintained, along with later additions, despite the tensions that led the country into a catastrophic civil war? To understand America’s identity as a nation, we will draw heavily upon the ideas of nationality discussed in the article '\textit{What is a Nation?}' by Stephen Brown. In particular, in this article, Brown asserts that a nationality is unified through “race, language, custom, religion, common interests, history and the men who made it, [or] a national government.” \autocite[]{10.2307/30083977} 

Looking over this list, we note that, in the revolutionary era, the people of the thirteen colonies shared a language, a history, and a racial identity (at least among ‘political’ citizens). Most were also followers of various denominations of the protestant faith. Yet, these things they also had in common with the British people, and indeed this is much of what defined the British nation, holding it together under the crown. So, language, race, religion, and shared history could not have been what compelled the colonies in their quest for independence. In fact, these would be the qualities that kept the colonists loyal to the king, even as they felt that their liberties were repeatedly violated: the Declaration of Rights and Grievances made by the Stamp Act Congress as well as addresses made to the king by the first Continental Congress stressed their continued allegiance to the crown. \autocite[pp. 172-173, 158-159]{AmericanRep1} 

Instead, we must look to these people’s common interests, their desire for a new set of customs, as well as their coming together under the leadership of the Continental Congress, to explain the inception of this new nation. The gentry in the colonies had a common interest in nullifying the taxation and trade regulation acts that they felt were unjustly imposed by parliament, these included the Stamp Act, the Sugar Act, and the Currency Act.\autocite[pp. 158-160]{AmericanRep1} Partly through the conscious efforts of the gentry, the middle class, artisans, laborers and shopkeepers, came to feel  that they too had an interest in this dispute, although they were predisposed to antagonizing the law makers in London, since the Quartering Act and ongoing impressments had left them embittered. Furthermore, policies such as the Navigations Act and the Currency Act financially pressed not only the upper classes but the entire free population of the colonies. The issues of principle in taxation without representation raised by lawyers such as Patrick Henry and John Dickinson simply rationalised the people’s feeling that they were the subject of injustice. So, the free people of the colonies wished to resist and to ultimately nullify the harsh economic policies enacted by parliament.  

Amongst this urge to resist the authority presiding over them from all the way across the Atlantic, the colonies found themselves more empathetic to the ideas of enlightenment thinkers. They wished not only to stop the current infringements on their liberties, but to live in a society where such transgressions by the government would never be tolerated. The second Continental Congress came together and led the colonies, during a time of crisis, towards this end, signified by the Declaration of Independence. Still prior to the declaration, however, they engaged in foreign diplomacy, raising of armed forces, the making of laws, and other matters of internal governance, essentially acting as a national government. \autocite[pp. 176]{AmericanRep1} The birth of the nation resulted from an alignment of interests, the pursuit of ideals in agreement with these interests, and the zeal of leaders who seized the moment. 

Of course, one of the key issues facing the new nation was that of slavery. In the 1850’s, while the hypocrisy of a slave holding nation built on ideals of liberty and the philosophies of the enlightenment was not lost on Americans, most Northerners found themselves capable of tolerating the institution of slavery (and reaping its benefits), at least as long as it was confined to the Southern states. Eventually, the South’s trademark institution, having grown in size and profitability throughout the nineteenth century, came to define the economy of the deep South states stretching from South Carolina to Texas. However, the plans of the newly formed, and overwhelmingly Northern, Republican party for the development of the western territories was at odds with the Southern plantation owners’ demands for more land and new soil. \autocite{EconomicOrig} This time the union could not withstand the tension, leading to the secession crisis of 1860-61. So, perhaps instead of asking 'What held the nation together?’, it is more appropriate to ask, 'What tore the nation apart?’.  

Certainly, the issue of race, now inseparable from the Southern institution of slavery, played a key role. “[Slavery] constituted a peculiar and powerful interest. All knew that this interest was, somehow, the cause of the war. To strengthen, perpetuate, and extend this interest was the object for which the insurgents would rend the Union, even by war.” \autocite[]{Lincoln}Alexander Stephens echoes this sentiment in his cornerstone speech. \autocite[]{Stephens} Yet, while the supremacy of the white man may have been a cornerstone of the Confederate constitution, and defense of this principle drove many a Confederate soldier, racial equality, or even abolition, was not the  prime directive for most of their counterparts in the Union army. \autocite[pp. 47-69]{WhatFoughtFor} Most were concerned first with preserving the nation as one, at least during the early stages of the war. Furthermore, what drove the Republican party into conflict with Southern states was not (primarily) their opposition to slavery on the basis of egalitarian ideals, but that the South’s vision of vast slave cotton plantations in the west ran contradictory to the Republican goal of preserving those plains for white settlers. As well, the Republicans were fierce supporters of federally financed infrastructure projects, especially in the North and North Western states and along the great lakes. The leaders of the Confederacy could not see any justice in Southerners bearing the burden for such projects.\autocite[]{Stephens}  

It is true however, that many Northerners, including Lincoln himself, recognized that the disunion which had brought about the war could only ever truly be remedied once slavery had been abolished in the nation altogether.\autocite[]{HouseDivided, WhatFoughtFor} Throughout the war, and especially after the Emancipation Proclamation brought the issue into focus, attitudes towards emancipation in the Union army shifted. A young private in the Union army, who had once written back home than he would never “Sacrifise [his life] for the liberty of a miserable black race of beings”, had come to rejoice in the eminent abolition of slavery towards the end of the war: “ Free free free yes free from that blighting curs Slavery the cause of four years of bloody warfare.” \footnote{As quoted in \textit{What They Fought For} by McPherson page 65. Could not find primary source.} 

Here again it seems that the initial decider of allegiances is the alignment of interests. For the Northerners, their economic and political development would be furthered by a position of antislavery. The South, especially the Deep South, saw the propagation of slavery as a necessity on the road to financial growth and social stability. However, how these interests sparked, and guided, the conversation on each side regarding the rights to liberty and equality molded the philosophies of both and hence much of what unified each trench. The South took a non-equivocal stance to white supremacy in no small part because it put the institution of African slavery, which the Confederate constitution explicitly protected,\footnote{See for example \textit{Constitution of the Confederate States} Article IV Sec. 3, available at \url{https://avalon.law.yale.edu/19th_century/csa_csa.asp#a3} (2020/06/16)} on firm ideological ground. On the other front, the Emancipation Proclamation ignited  debates and discussions amongst Northerners, which eventually convinced many who were previously against abolition or on the fence, that at the very least it was a practical necessity for the U.S. constitution to fulfill its promises to men of all colors.  

During the Reconstruction era following the Civil War, it became clear that while the Confederacy had laid down their arms, Southern loyalties still lay with the ideologies that had started the war. As soon as they were given the chance under President Andrew Johnson, the Southern states bolstered ex-Confederates back into power and adopted the so called ‘Black Code’ laws; making it unequivocally clear that they equated black freedom with black serfdom. The Federal government intervened to maintain public order and enact the changes that the South would not submit to willingly. Congress passed the Civil Rights Act of 1866 and the Freedman’s Bureau Bill, both against President Johnson's veto, followed by the ratification of the Fourteenth Amendment. This pattern of Southern resistance to change followed by Federal interference continued well into the 1870s.\autocite[pp. 531-559]{AmericanRep1} Here again, as it had been during the war, it was the Federal government’s power to carry the states along by force that brought any sense of real cohesion to the fragile tapestry that was the North and South. Had Congress remained idle as Confederate leaders regained power, with little change in their vision for the states, then there would have been little practical difference to life before and after the war for many Southerners.  

While the inception of the union among the States has (perhaps deservedly) a halo of humanist and liberal ideals, these ideals have often been interpreted through the lens of regional politico-economic interests, rather than the reverse. These observation and others make it abundantly clear that what truly kept the nation together, or caused discord, during the era of independence and in the years since has been the common interests of its inhabitants as well as the power of the Federal government to forcefully resolve any division caused when these interests collide. Though the exercise of these powers has often not been entirely successful, it has been necessary for the nation to reach a resolution to its internal conflicts while still intact.  


\printbibliography
 

 


\end{document}